This chapter introduces key folders and files in the MMB package. The content of key files is described briefly in this chapter. For details read chapter \ref{sec:ModelfileStructure}. 

\subsection*{Key Folders}
 
Users can find three key folders under the directory \textit{’resources/app/dist/electron/static/mmci- cli’}. The folder \textit{MODELS} includes a list of folders for each model in the MMB, with one mod-file and one json-file that links the mod-file and the MMB interface inside. The folder \textit{RULES} includes json-files for all the common monetary policy rules in the MMB. The subfolder \textit{ALTOOLS} in the folder \textit{LIB} includes scripts for adaptive learning models.
\subsection*{Some key files}
In earlier versions, the MMB had the subfolder \textit{MMB\_OPTIONS} containing specific MATLAB files related to the usage of the Modelbase for policy analysis and comparison, as well as explanatory notes for models and policy rules. These folders have been removed. The information is mostly contained in the .json file in the models folder.   

The files discussed here are stored in the folder \textit{MODELS} and are a brief introduction. For further information read chapter \ref{sec:ModelfileStructure}.

\begin{itemize}
\item \textbf{modelname.mod}\\
This mod-file contains the model variables, shocks and parameters as well as the model equations for economies with flexible or sticky price and wage. If the model is nonlinear, the file should also include an \textit{initval} block, which provides guess values for non-linear solvers and steady state computations.
\item \textbf{modelname.json}\\
The json-file illustrates a model’s capabilities, including which monetary policy rules and shocks are applicable to the model, whether or not unconditional variances could be calculated, and whether or not the model features adaptive learning. If the latter is true, then the json-file shows which variables are forwards and states (divided into states short and long, depending on the adaptive learning mechanism). Parameters of the model’s specific policy rule, if available, is also specified in the json-file. Users can also change a variable’s or shocks' notation in the json-file.
\item \textbf{optional: modelname\_steadystate.m}\\
The m-file contains the steady state values and equations in case of non-zero steady states. 
\end{itemize}


